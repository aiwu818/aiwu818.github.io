%%% XeLatex, TeXLive
\documentclass[a4paper, 11pt]{article}
\usepackage{comment} % enables the use of multi-line comments (\ifx \fi) 
\usepackage{lipsum} %This package just generates Lorem Ipsum filler text. 
\usepackage{fullpage} % changes the margin
\usepackage[a4paper, total={7in, 10in}]{geometry}
\usepackage[fleqn]{amsmath}
\usepackage{amssymb,amsthm}  % assumes amsmath package installed
\newtheorem{theorem}{Theorem}
\newtheorem{corollary}{Corollary}
\usepackage{graphicx}
\usepackage{tikz}
\usetikzlibrary{arrows}
\usepackage{verbatim}
%\usepackage[numbered]{mcode}
\usepackage{float}
\usepackage{tikz}
    \usetikzlibrary{shapes,arrows}
    \usetikzlibrary{arrows,calc,positioning}
    \tikzset{
        block/.style = {draw, rectangle,
            minimum height=1cm,
            minimum width=1.5cm},
        input/.style = {coordinate,node distance=1cm},
        output/.style = {coordinate,node distance=4cm},
        arrow/.style={draw, -latex,node distance=2cm},
        pinstyle/.style = {pin edge={latex-, black,node distance=2cm}},
        sum/.style = {draw, circle, node distance=1cm},
    }
\usepackage{xcolor}
\usepackage{mdframed}
\usepackage[shortlabels]{enumitem}
\usepackage{indentfirst}
\usepackage{hyperref}
\renewcommand{\thesubsection}{\thesection.\alph{subsection}}
\newenvironment{problem}[2][练习题]
    { \begin{mdframed}[backgroundcolor=gray!5] \textbf{#1 #2} \\}
    {  \end{mdframed}}
% Define solution environment
\newenvironment{solution}
    {\textbf{解:}}
    {}
\renewcommand{\qed}{\quad\qedsymbol}
% Define solution environment
\newenvironment{solution2}
    {\textbf{证明:}}
    {}
\renewcommand{\qed}{\quad\qedsymbol}
%\usepackage[utf8]{ctex}
\usepackage[punct]{ctex}


%%%%%%%%%%%%%%%%%%%%%%%%%%%%%%%%%%%%%%%%%%%%%%%%%%%%%%%%%%%%%%%%%%%%%%%%%%%%%%%%%%%%%%%%%%%%%%%%%%%%%%%%%%%%%%%%%%%%%%%%%%%%%%%%%%%%%%%%
\begin{document}
%Header-Make sure you update this information!!!!
\noindent
%%%%%%%%%%%%%%%%%%%%%%%%%%%%%%%%%%%%%%%%%%%%%%%%%%%%%%%%%%%%%%%%%%%%%%%%%%%%%%%%%%%%%%%%%%%%%%%%%%%%%%%%%%%%%%%%%%%%%%%%%%%%%%%%%%%%%%%%
\large\textbf{3212090710***\quad 姓名} \hfill \textbf{《高等代数选讲》作业 - No.~\Huge{08}}   \\
专业: 信息与计算科学 \hfill 学号: 3212090710*** \\
\normalsize 课程名称: 385460 - 高等代数选讲 (Selection of Advanced Algebra) \hfill 学期: 2024 春 (Spring 2024)\\
授课教师: 艾武(教授) \hfill 完成时间: 2024年04月29日 \\
\noindent\rule{7in}{1.8pt}

%%%%%%%%%%%%%%%%%%%%%%%%%%%%%%%%%%%%%%%%%%%%%%%%%%%%%%%%%%%%%%%%%%%%%%%%%%%%%%%%%%%%%%%%%%%%%%%%%%%%%%%%%%%%%%%%%%%%%%%%%%%%%%%%%%%%%%%%
% Problem 1
%%%%%%%%%%%%%%%%%%%%%%%%%%%%%%%%%%%%%%%%%%%%%%%%%%%%%%%%%%%%%%%%%%%%%%%%%%%%%%%%%%%%%%%%%%%%%%%%%%%%%%%%%%%%%%%%%%%%%%%%%%%%%%%%%%%%%%%%
\begin{problem}{1}
\textbf{(高等代数,郑州大学,2024年)}: 设 $V=P^{2 \times 2}$ 是数域 $P$ 上的线性空间, $A=\left(\begin{array}{ll}a & b \\ c & d\end{array}\right) \in V$, 线性变换 $\sigma: X \rightarrow A X, \forall X \in P^{2 \times 2}$.

(1)求线性变换 $\sigma$ 在基
$$
E_{11}=\left(\begin{array}{ll}
1 & 0 \\
0 & 0
\end{array}\right), E_{12}=\left(\begin{array}{ll}
0 & 1 \\
0 & 0
\end{array}\right), E_{21}=\left(\begin{array}{ll}
0 & 0 \\
1 & 0
\end{array}\right), E_{22}=\left(\begin{array}{ll}
0 & 0 \\
0 & 1
\end{array}\right)
$$
下的矩阵.

(2) 如果 $A$ 相似于对角矩阵, 证明线性变换 $\sigma$ 在 $V$ 的某组基下的矩阵是对角矩阵.
\end{problem}

\begin{solution}
%% 过程写这里

\end{solution} 

%\noindent\rule{7in}{2.8pt} 
%%%%%%%%%%%%%%%%%%%%%%%%%%%%%%%%%%%%%%%%%%%%%%%%%%%%%%%%%%%%%%%%%%%%%%%%%
% Problem 2
%%%%%%%%%%%%%%%%%%%%%%%%%%%%%%%%%%%%%%%%%%%%%%%%%%%%%%%%%%%%%%%%%%%%%%%%%%%%%%%%%%%%%%%%%%%%%%%%%%%%%%%%%%%%%%%%%%%%%%%%%%%%%%%%%%%%%%%%

\begin{problem}{2}
\textbf{(线性代数与解析几何,中国科学技术大学,2024年)}: 对于任意向量 $X, Y \in \mathbb{R}^3$, 定义
$$
(X, Y)=X^T\left(\begin{array}{ccc}
2 & 1 & 1 \\
1 & 2 & 1 \\
1 & 1 & 2
\end{array}\right) Y .
$$

(1)证明上述定义给出了 $\mathbb{R}^3$ 上的一个内积.

(2)利用 Schmidt 正交化从基向量组 $\alpha_1=(1,0,0)^T, \alpha_2=(1,0,1)^T, \alpha_3=(1,1,0)^T$ 构造一组标准正交基.
\end{problem}

\begin{solution}
%% 过程写这里


\end{solution} 





%%%%%%%%%%%%%%%%%%%%%%%%%%%%%%%%%%%%%%%%%%%%%%%%%%%%%%%%%%%%%%%%%%%%%%%%%
\end{document}
