%%% XeLatex, TeXLive
\documentclass[a4paper, 11pt]{article}
\usepackage{comment} % enables the use of multi-line comments (\ifx \fi) 
\usepackage{lipsum} %This package just generates Lorem Ipsum filler text. 
\usepackage{fullpage} % changes the margin
\usepackage[a4paper, total={7in, 10in}]{geometry}
\usepackage[fleqn]{amsmath}
\usepackage{amssymb,amsthm}  % assumes amsmath package installed
\newtheorem{theorem}{Theorem}
\newtheorem{corollary}{Corollary}
\usepackage{graphicx}
\usepackage{tikz}
\usetikzlibrary{arrows}
\usepackage{verbatim}
%\usepackage[numbered]{mcode}
\usepackage{float}
\usepackage{tikz}
    \usetikzlibrary{shapes,arrows}
    \usetikzlibrary{arrows,calc,positioning}
    \tikzset{
        block/.style = {draw, rectangle,
            minimum height=1cm,
            minimum width=1.5cm},
        input/.style = {coordinate,node distance=1cm},
        output/.style = {coordinate,node distance=4cm},
        arrow/.style={draw, -latex,node distance=2cm},
        pinstyle/.style = {pin edge={latex-, black,node distance=2cm}},
        sum/.style = {draw, circle, node distance=1cm},
    }
\usepackage{xcolor}
\usepackage{mdframed}
\usepackage[shortlabels]{enumitem}
\usepackage{indentfirst}
\usepackage{hyperref}
\renewcommand{\thesubsection}{\thesection.\alph{subsection}}
\newenvironment{problem}[2][练习题]
    { \begin{mdframed}[backgroundcolor=gray!5] \textbf{#1 #2} \\}
    {  \end{mdframed}}
% Define solution environment
\newenvironment{solution}
    {\textbf{解:}}
    {}
\renewcommand{\qed}{\quad\qedsymbol}
% Define solution environment
\newenvironment{solution2}
    {\textbf{证明:}}
    {}
\renewcommand{\qed}{\quad\qedsymbol}
%\usepackage[utf8]{ctex}
\usepackage[punct]{ctex}


%%%%%%%%%%%%%%%%%%%%%%%%%%%%%%%%%%%%%%%%%%%%%%%%%%%%%%%%%%%%%%%%%%%%%%%%%%%%%%%%%%%%%%%%%%%%%%%%%%%%%%%%%%%%%%%%%%%%%%%%%%%%%%%%%%%%%%%%
\begin{document}
%Header-Make sure you update this information!!!!
\noindent
%%%%%%%%%%%%%%%%%%%%%%%%%%%%%%%%%%%%%%%%%%%%%%%%%%%%%%%%%%%%%%%%%%%%%%%%%%%%%%%%%%%%%%%%%%%%%%%%%%%%%%%%%%%%%%%%%%%%%%%%%%%%%%%%%%%%%%%%
\large\textbf{3212090710***\quad 姓名} \hfill \textbf{《高等代数选讲》作业 - No.~\Huge{07}}   \\
专业: 信息与计算科学 \hfill 学号: 3212090710*** \\
\normalsize 课程名称: 385460 - 高等代数选讲 (Selection of Advanced Algebra) \hfill 学期: 2024 春 (Spring 2024)\\
授课教师: 艾武(教授) \hfill 完成时间: 2024年04月22日 \\
\noindent\rule{7in}{1.8pt}

%%%%%%%%%%%%%%%%%%%%%%%%%%%%%%%%%%%%%%%%%%%%%%%%%%%%%%%%%%%%%%%%%%%%%%%%%%%%%%%%%%%%%%%%%%%%%%%%%%%%%%%%%%%%%%%%%%%%%%%%%%%%%%%%%%%%%%%%
% Problem 1
%%%%%%%%%%%%%%%%%%%%%%%%%%%%%%%%%%%%%%%%%%%%%%%%%%%%%%%%%%%%%%%%%%%%%%%%%%%%%%%%%%%%%%%%%%%%%%%%%%%%%%%%%%%%%%%%%%%%%%%%%%%%%%%%%%%%%%%%
\begin{problem}{1}
\textbf{(高等代数,南开大学,2024年)}: 设 $V=\mathbb{R}[x]_4$ 为次数小于 4 的实系数多项式构成的实线性空间, 定义线性变换 $\sigma: V \rightarrow V$ 如下:
$$
\sigma(f(x))=\left(1-x^2\right) f^{\prime \prime}(x)-2 x f^{\prime}(x), \forall f(x) \in V .
$$
求 $V$ 的一组基, 设 $\sigma$ 在这组基下的矩阵为对角矩阵.

\end{problem}

\begin{solution}
%% 过程写这里

\end{solution} 

%\noindent\rule{7in}{2.8pt} 
%%%%%%%%%%%%%%%%%%%%%%%%%%%%%%%%%%%%%%%%%%%%%%%%%%%%%%%%%%%%%%%%%%%%%%%%%
% Problem 2
%%%%%%%%%%%%%%%%%%%%%%%%%%%%%%%%%%%%%%%%%%%%%%%%%%%%%%%%%%%%%%%%%%%%%%%%%%%%%%%%%%%%%%%%%%%%%%%%%%%%%%%%%%%%%%%%%%%%%%%%%%%%%%%%%%%%%%%%

\begin{problem}{2}
\textbf{(高等代数,南京师范大学,2024年)}: 设 $V$ 是全体次数不超过 $n$ 的实系数多项式再添上零多项式组成的实数域上的线性空间, 定义$\sigma$ 是 $V$ 上的线性变换: 任给 $f(x) \in V, \sigma(f(x))=x f^{\prime}(x)-f(x)$.

(1) 求 $\sigma$ 的核 $\sigma^{-1}(0)$ 及 $\sigma$ 的值域 $\sigma V$.

(2) 证明: $V=\sigma^{-1}(0) \oplus \sigma V$.
\end{problem}

\begin{solution}
%% 过程写这里


\end{solution} 





%%%%%%%%%%%%%%%%%%%%%%%%%%%%%%%%%%%%%%%%%%%%%%%%%%%%%%%%%%%%%%%%%%%%%%%%%
\end{document}
