%%% XeLatex, TeXLive
\documentclass[a4paper, 11pt]{article}
\usepackage{comment} % enables the use of multi-line comments (\ifx \fi) 
\usepackage{lipsum} %This package just generates Lorem Ipsum filler text. 
\usepackage{fullpage} % changes the margin
\usepackage[a4paper, total={7in, 10in}]{geometry}
\usepackage[fleqn]{amsmath}
\usepackage{amssymb,amsthm}  % assumes amsmath package installed
\newtheorem{theorem}{Theorem}
\newtheorem{corollary}{Corollary}
\usepackage{graphicx}
\usepackage{tikz}
\usetikzlibrary{arrows}
\usepackage{verbatim}
%\usepackage[numbered]{mcode}
\usepackage{float}
\usepackage{tikz}
    \usetikzlibrary{shapes,arrows}
    \usetikzlibrary{arrows,calc,positioning}
    \tikzset{
        block/.style = {draw, rectangle,
            minimum height=1cm,
            minimum width=1.5cm},
        input/.style = {coordinate,node distance=1cm},
        output/.style = {coordinate,node distance=4cm},
        arrow/.style={draw, -latex,node distance=2cm},
        pinstyle/.style = {pin edge={latex-, black,node distance=2cm}},
        sum/.style = {draw, circle, node distance=1cm},
    }
\usepackage{xcolor}
\usepackage{mdframed}
\usepackage[shortlabels]{enumitem}
\usepackage{indentfirst}
\usepackage{hyperref}
\renewcommand{\thesubsection}{\thesection.\alph{subsection}}
\newenvironment{problem}[2][练习题]
    { \begin{mdframed}[backgroundcolor=gray!5] \textbf{#1 #2} \\}
    {  \end{mdframed}}
% Define solution environment
\newenvironment{solution}
    {\textbf{解:}}
    {}
\renewcommand{\qed}{\quad\qedsymbol}
% Define solution environment
\newenvironment{solution2}
    {\textbf{证明:}}
    {}
\renewcommand{\qed}{\quad\qedsymbol}
%\usepackage[utf8]{ctex}
\usepackage[punct]{ctex}


%%%%%%%%%%%%%%%%%%%%%%%%%%%%%%%%%%%%%%%%%%%%%%%%%%%%%%%%%%%%%%%%%%%%%%%%%%%%%%%%%%%%%%%%%%%%%%%%%%%%%%%%%%%%%%%%%%%%%%%%%%%%%%%%%%%%%%%%
\begin{document}
%Header-Make sure you update this information!!!!
\noindent
%%%%%%%%%%%%%%%%%%%%%%%%%%%%%%%%%%%%%%%%%%%%%%%%%%%%%%%%%%%%%%%%%%%%%%%%%%%%%%%%%%%%%%%%%%%%%%%%%%%%%%%%%%%%%%%%%%%%%%%%%%%%%%%%%%%%%%%%
\large\textbf{3212090710***\quad 姓名} \hfill \textbf{《高等代数选讲》作业 - No.~\Huge{03}}   \\
专业: 信息与计算科学 \hfill 学号: 3212090710*** \\
\normalsize 课程名称: 385460 - 高等代数选讲 (Selection of Advanced Algebra) \hfill 学期: 2024 春 (Spring 2024)\\
授课教师: 艾武(教授) \hfill 完成时间: 2024年03月25日 \\
\noindent\rule{7in}{1.8pt}

%%%%%%%%%%%%%%%%%%%%%%%%%%%%%%%%%%%%%%%%%%%%%%%%%%%%%%%%%%%%%%%%%%%%%%%%%%%%%%%%%%%%%%%%%%%%%%%%%%%%%%%%%%%%%%%%%%%%%%%%%%%%%%%%%%%%%%%%
% Problem 1
%%%%%%%%%%%%%%%%%%%%%%%%%%%%%%%%%%%%%%%%%%%%%%%%%%%%%%%%%%%%%%%%%%%%%%%%%%%%%%%%%%%%%%%%%%%%%%%%%%%%%%%%%%%%%%%%%%%%%%%%%%%%%%%%%%%%%%%%
\begin{problem}{1}
\textbf{(高等代数,重庆大学,2024年)}: 设线性方程组
$$
\left\{\begin{array}{l}
k x_1+x_2+x_3=k-3 ; \\
x_1+k x_2+x_3=-2 ; \\
x_1+x_2+k x_3=-2 .
\end{array}\right.
$$

(1) $k$ 为何值时, 方程组无解?

(2) $k$ 为何值时, 方程组存在唯一解?

(3) $k$ 为何值时, 方程组存在无穷多解? 并求出通解.
\end{problem}

\begin{solution}
%% 过程写这里

\end{solution} 

%\noindent\rule{7in}{2.8pt} 
%%%%%%%%%%%%%%%%%%%%%%%%%%%%%%%%%%%%%%%%%%%%%%%%%%%%%%%%%%%%%%%%%%%%%%%%%
% Problem 2
%%%%%%%%%%%%%%%%%%%%%%%%%%%%%%%%%%%%%%%%%%%%%%%%%%%%%%%%%%%%%%%%%%%%%%%%%%%%%%%%%%%%%%%%%%%%%%%%%%%%%%%%%%%%%%%%%%%%%%%%%%%%%%%%%%%%%%%%

\begin{problem}{2}
\textbf{(高等代数,南昌大学,2024年)}: 设 $\alpha_i=\left(a_{i 1}, a_{i 2}, \cdots, a_{i n}\right)(i=1,2, \cdots, s), \beta=\left(b_1, b_2, \cdots, b_n\right)$. 证明: 若线性方程组
$$
\left\{\begin{array}{c}
a_{11} x_1+a_{12} x_2+\cdots+a_{1 n} x_n=0 ; \\
a_{21} x_1+a_{22} x_2+\cdots+a_{2 n} x_n=0 ; \\
\cdots \cdots \\
a_{s 1} x_1+a_{s 2} x_2+\cdots+a_{s n} x_n=0 .
\end{array}\right.
$$
的解全是 $b_1 x_1+b_2 x_2+\cdots+b_n x_n=0$ 的解, 那么 $\beta$ 可经 $\alpha_1, \alpha_2, \cdots, \alpha_s$ 线性表出.
\end{problem}

\begin{solution2}
%% 过程写这里


\end{solution2} 





%%%%%%%%%%%%%%%%%%%%%%%%%%%%%%%%%%%%%%%%%%%%%%%%%%%%%%%%%%%%%%%%%%%%%%%%%
\end{document}
