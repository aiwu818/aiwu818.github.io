%%% XeLatex, TeXLive
\documentclass[a4paper, 11pt]{article}
\usepackage{comment} % enables the use of multi-line comments (\ifx \fi) 
\usepackage{lipsum} %This package just generates Lorem Ipsum filler text. 
\usepackage{fullpage} % changes the margin
\usepackage[a4paper, total={7in, 10in}]{geometry}
\usepackage[fleqn]{amsmath}
\usepackage{amssymb,amsthm}  % assumes amsmath package installed
\newtheorem{theorem}{Theorem}
\newtheorem{corollary}{Corollary}
\usepackage{graphicx}
\usepackage{tikz}
\usetikzlibrary{arrows}
\usepackage{verbatim}
%\usepackage[numbered]{mcode}
\usepackage{float}
\usepackage{tikz}
    \usetikzlibrary{shapes,arrows}
    \usetikzlibrary{arrows,calc,positioning}
    \tikzset{
        block/.style = {draw, rectangle,
            minimum height=1cm,
            minimum width=1.5cm},
        input/.style = {coordinate,node distance=1cm},
        output/.style = {coordinate,node distance=4cm},
        arrow/.style={draw, -latex,node distance=2cm},
        pinstyle/.style = {pin edge={latex-, black,node distance=2cm}},
        sum/.style = {draw, circle, node distance=1cm},
    }
\usepackage{xcolor}
\usepackage{mdframed}
\usepackage[shortlabels]{enumitem}
\usepackage{indentfirst}
\usepackage{hyperref}
\renewcommand{\thesubsection}{\thesection.\alph{subsection}}
\newenvironment{problem}[2][练习题]
    { \begin{mdframed}[backgroundcolor=gray!5] \textbf{#1 #2} \\}
    {  \end{mdframed}}
% Define solution environment
\newenvironment{solution}
    {\textbf{解:}}
    {}
\renewcommand{\qed}{\quad\qedsymbol}
% Define solution environment
\newenvironment{solution2}
    {\textbf{证明:}}
    {}
\renewcommand{\qed}{\quad\qedsymbol}
%\usepackage[utf8]{ctex}
\usepackage[punct]{ctex}


%%%%%%%%%%%%%%%%%%%%%%%%%%%%%%%%%%%%%%%%%%%%%%%%%%%%%%%%%%%%%%%%%%%%%%%%%%%%%%%%%%%%%%%%%%%%%%%%%%%%%%%%%%%%%%%%%%%%%%%%%%%%%%%%%%%%%%%%
\begin{document}
%Header-Make sure you update this information!!!!
\noindent
%%%%%%%%%%%%%%%%%%%%%%%%%%%%%%%%%%%%%%%%%%%%%%%%%%%%%%%%%%%%%%%%%%%%%%%%%%%%%%%%%%%%%%%%%%%%%%%%%%%%%%%%%%%%%%%%%%%%%%%%%%%%%%%%%%%%%%%%
\large\textbf{3212090710***\quad 姓名} \hfill \textbf{《高等代数选讲》作业 - No.~\Huge{04}}   \\
专业: 信息与计算科学 \hfill 学号: 3212090710*** \\
\normalsize 课程名称: 385460 - 高等代数选讲 (Selection of Advanced Algebra) \hfill 学期: 2024 春 (Spring 2024)\\
授课教师: 艾武(教授) \hfill 完成时间: 2024年04月01日 \\
\noindent\rule{7in}{1.8pt}

%%%%%%%%%%%%%%%%%%%%%%%%%%%%%%%%%%%%%%%%%%%%%%%%%%%%%%%%%%%%%%%%%%%%%%%%%%%%%%%%%%%%%%%%%%%%%%%%%%%%%%%%%%%%%%%%%%%%%%%%%%%%%%%%%%%%%%%%
% Problem 1
%%%%%%%%%%%%%%%%%%%%%%%%%%%%%%%%%%%%%%%%%%%%%%%%%%%%%%%%%%%%%%%%%%%%%%%%%%%%%%%%%%%%%%%%%%%%%%%%%%%%%%%%%%%%%%%%%%%%%%%%%%%%%%%%%%%%%%%%
\begin{problem}{1}
\textbf{(线性代数与常微分方程,山东大学,2024年)}: 设 $A, B, C$ 分别为 $m \times n, n \times t, s \times m$ 阶矩阵.

(1) 若矩阵 $A$ 的秩 $r(A)=r$, 证明: 存在可逆阵 $P, Q$, 使得 $P A$ 的后 $m-r$ 行全为零, $A Q$ 的后 $n-r$ 列全为零.

(2) 利用 (1) 证明: 若 $r(A)=n$, 则 $r(A B)=r(B)$; 若 $r(A)=m$, 则 $r(C A)=r(C)$.
\end{problem}

\begin{solution2}
%% 过程写这里

\end{solution2} 

%\noindent\rule{7in}{2.8pt} 
%%%%%%%%%%%%%%%%%%%%%%%%%%%%%%%%%%%%%%%%%%%%%%%%%%%%%%%%%%%%%%%%%%%%%%%%%
% Problem 2
%%%%%%%%%%%%%%%%%%%%%%%%%%%%%%%%%%%%%%%%%%%%%%%%%%%%%%%%%%%%%%%%%%%%%%%%%%%%%%%%%%%%%%%%%%%%%%%%%%%%%%%%%%%%%%%%%%%%%%%%%%%%%%%%%%%%%%%%

\begin{problem}{2}
\textbf{(高等代数,南京理工大学,2024年)}: 设 $A$ 为一个秩为 $r$ 的 $n$ 阶方阵, 且 $A^2=A$. 证明: 存在一个秩为 $r$ 的 $r \times n$ 矩阵 $B$ 与一个秩为 $r$ 的 $n \times r$ 矩阵 $C$, 满足 $A=C B$, 且 $B C=E_r$ ( $E_r$ 为 $r$ 阶单位矩阵).
\end{problem}

\begin{solution2}
%% 过程写这里


\end{solution2} 





%%%%%%%%%%%%%%%%%%%%%%%%%%%%%%%%%%%%%%%%%%%%%%%%%%%%%%%%%%%%%%%%%%%%%%%%%
\end{document}
