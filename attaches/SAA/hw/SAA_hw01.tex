%%% XeLatex, TeXLive
\documentclass[a4paper, 11pt]{article}
\usepackage{comment} % enables the use of multi-line comments (\ifx \fi) 
\usepackage{lipsum} %This package just generates Lorem Ipsum filler text. 
\usepackage{fullpage} % changes the margin
\usepackage[a4paper, total={7in, 10in}]{geometry}
\usepackage[fleqn]{amsmath}
\usepackage{amssymb,amsthm}  % assumes amsmath package installed
\newtheorem{theorem}{Theorem}
\newtheorem{corollary}{Corollary}
\usepackage{graphicx}
\usepackage{tikz}
\usetikzlibrary{arrows}
\usepackage{verbatim}
%\usepackage[numbered]{mcode}
\usepackage{float}
\usepackage{tikz}
    \usetikzlibrary{shapes,arrows}
    \usetikzlibrary{arrows,calc,positioning}
    \tikzset{
        block/.style = {draw, rectangle,
            minimum height=1cm,
            minimum width=1.5cm},
        input/.style = {coordinate,node distance=1cm},
        output/.style = {coordinate,node distance=4cm},
        arrow/.style={draw, -latex,node distance=2cm},
        pinstyle/.style = {pin edge={latex-, black,node distance=2cm}},
        sum/.style = {draw, circle, node distance=1cm},
    }
\usepackage{xcolor}
\usepackage{mdframed}
\usepackage[shortlabels]{enumitem}
\usepackage{indentfirst}
\usepackage{hyperref}
\renewcommand{\thesubsection}{\thesection.\alph{subsection}}
\newenvironment{problem}[2][练习题]
    { \begin{mdframed}[backgroundcolor=gray!5] \textbf{#1 #2} \\}
    {  \end{mdframed}}
% Define solution environment
\newenvironment{solution}
    {\textbf{解:}}
    {}
\renewcommand{\qed}{\quad\qedsymbol}
%\usepackage[utf8]{ctex}
\usepackage[punct]{ctex}


%%%%%%%%%%%%%%%%%%%%%%%%%%%%%%%%%%%%%%%%%%%%%%%%%%%%%%%%%%%%%%%%%%%%%%%%%%%%%%%%%%%%%%%%%%%%%%%%%%%%%%%%%%%%%%%%%%%%%%%%%%%%%%%%%%%%%%%%
\begin{document}
%Header-Make sure you update this information!!!!
\noindent
%%%%%%%%%%%%%%%%%%%%%%%%%%%%%%%%%%%%%%%%%%%%%%%%%%%%%%%%%%%%%%%%%%%%%%%%%%%%%%%%%%%%%%%%%%%%%%%%%%%%%%%%%%%%%%%%%%%%%%%%%%%%%%%%%%%%%%%%
\large\textbf{3212090710***\quad 姓名} \hfill \textbf{《高等代数选讲》作业 - No.~\Huge{01}}   \\
专业: 信息与计算科学 \hfill 学号: 3212090710*** \\
\normalsize 课程名称: 385460 - 高等代数选讲 (Selection of Advanced Algebra) \hfill 学期: 2024 春 (Spring 2024)\\
授课教师: 艾武(教授) \hfill 完成时间: 2024年03月11日 \\
\noindent\rule{7in}{1.8pt}

%%%%%%%%%%%%%%%%%%%%%%%%%%%%%%%%%%%%%%%%%%%%%%%%%%%%%%%%%%%%%%%%%%%%%%%%%%%%%%%%%%%%%%%%%%%%%%%%%%%%%%%%%%%%%%%%%%%%%%%%%%%%%%%%%%%%%%%%
% Problem 1
%%%%%%%%%%%%%%%%%%%%%%%%%%%%%%%%%%%%%%%%%%%%%%%%%%%%%%%%%%%%%%%%%%%%%%%%%%%%%%%%%%%%%%%%%%%%%%%%%%%%%%%%%%%%%%%%%%%%%%%%%%%%%%%%%%%%%%%%
\begin{problem}{1}
\textbf{(高等代数,南开大学,2022年)}: 计算行列式:
\begin{align*}
D_{4} & = \left|\begin{array}{llll}
2^{4}+1 & 2^{3} & 2^{2} & 2 \\
3^{4}+1 & 3^{3} & 3^{2} & 3 \\
4^{4}+1 & 4^{3} & 4^{2} & 4 \\
5^{4}+1 & 5^{3} & 5^{2} & 5
\end{array}\right| .
\end{align*}
\end{problem}

\begin{solution}
\begin{align*}
D_{4} & = \left|\begin{array}{llll}
2^{4}+1 & 2^{3} & 2^{2} & 2 \\
3^{4}+1 & 3^{3} & 3^{2} & 3 \\
4^{4}+1 & 4^{3} & 4^{2} & 4 \\
5^{4}+1 & 5^{3} & 5^{2} & 5
\end{array}\right|
=
\end{align*}
\end{solution} 

%\noindent\rule{7in}{2.8pt}
%%%%%%%%%%%%%%%%%%%%%%%%%%%%%%%%%%%%%%%%%%%%%%%%%%%%%%%%%%%%%%%%%%%%%%%%%
% Problem 2
%%%%%%%%%%%%%%%%%%%%%%%%%%%%%%%%%%%%%%%%%%%%%%%%%%%%%%%%%%%%%%%%%%%%%%%%%%%%%%%%%%%%%%%%%%%%%%%%%%%%%%%%%%%%%%%%%%%%%%%%%%%%%%%%%%%%%%%%

\begin{problem}{2}
\textbf{(高等代数,中山大学,2022年)}: 计算行列式:
\begin{align*}
D_{5} & =\left|\begin{array}{ccccc}
2 & 1 & 1 & 1 & 1 \\
1 & \frac{3}{2} & 1 & 1 & 1 \\
1 & 1 & \frac{4}{3} & 1 & 1 \\
1 & 1 & 1 & \frac{5}{4} & 1 \\
1 & 1 & 1 & 1 & \frac{6}{5}
\end{array}\right|
\end{align*}
\end{problem}

\begin{solution}
\begin{align*}
D_{5} & =\left|\begin{array}{ccccc}
2 & 1 & 1 & 1 & 1 \\
1 & \frac{3}{2} & 1 & 1 & 1 \\
1 & 1 & \frac{4}{3} & 1 & 1 \\
1 & 1 & 1 & \frac{5}{4} & 1 \\
1 & 1 & 1 & 1 & \frac{6}{5}
\end{array}\right|
=
\end{align*}

\end{solution} 

%\noindent\rule{7in}{2.8pt}
%%%%%%%%%%%%%%%%%%%%%%%%%%%%%%%%%%%%%%%%%%%%%%%%%%%%%%%%%%%%%%%%%%%%%%%%%%%%%%%%%%%%%%%%%%%%%%%%%%%%%%%%%%%%%%%%%%%%%%%%%%%%%%%%%%%%%%%%
% Problem 3
%%%%%%%%%%%%%%%%%%%%%%%%%%%%%%%%%%%%%%%%%%%%%%%%%%%%%%%%%%%%%%%%%%%%%%%%%%%%%%%%%%%%%%%%%%%%%%%%%%%%%%%%%%%%%%%%%%%%%%%%%%%%%%%%%%%%%%%%
\begin{problem}{3}
\textbf{(高等代数与解析几何,东北师范大学,2024年)}: 求下列行列式:
\begin{align*}
\left|\begin{array}{cccc}
1+x_1 & 1+x_1^2 & \cdots & 1+x_1^n \\
1+x_2 & 1+x_2^2 & \cdots & 1+x_2^n \\
\vdots & \vdots & & \vdots \\
1+x_n & 1+x_n^2 & \cdots & 1+x_n^n
\end{array}\right|
\end{align*}
\end{problem}

\begin{solution}
\begin{align*}
\left|\begin{array}{cccc}
1+x_1 & 1+x_1^2 & \cdots & 1+x_1^n \\
1+x_2 & 1+x_2^2 & \cdots & 1+x_2^n \\
\vdots & \vdots & & \vdots \\
1+x_n & 1+x_n^2 & \cdots & 1+x_n^n
\end{array}\right|
=
\end{align*}
\end{solution} 



%%%%%%%%%%%%%%%%%%%%%%%%%%%%%%%%%%%%%%%%%%%%%%%%%%%%%%%%%%%%%%%%%%%%%%%%%
\end{document}
